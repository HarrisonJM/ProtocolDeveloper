\hypertarget{group__testCaseHandler}{}\section{Test\+Case\+Handler}
\label{group__testCaseHandler}\index{TestCaseHandler@{TestCaseHandler}}


Holds all the data\+Points.  


Holds all the data\+Points. 

Stores a single variable to be acted upon.

Returns information/data to be sent after the user requests it.

Defines a parser that can walk/traverse an X\+ML document tree.

Top-\/level class that returns a data emitter.

\char`\"{}\+Compiles\char`\"{} the X\+ML into a form that can create data to be used

Stores operations that a thread must take.

Stores information pertaining to a Datapoints desired operations.

An interpreted Data Point.

An entire datapoint.

Stores at least one data\+Point and emits the data\+Points data.

\begin{DoxyDate}{Date}
10/07/18
\end{DoxyDate}
Will need to be threadsafe so that multiple threads can access data items (and the next data items) without breaking.

Organises the variables and outputs as a std\+::string

\begin{DoxyDate}{Date}
04/\+Jul/2018
\end{DoxyDate}
Emits built data items to be sent out by tests. Threadsafe

\begin{DoxyDate}{Date}
4/\+Jul/2018
\end{DoxyDate}
Will be read by the protocol that will help describe what action to take should a certain condition be met

\begin{DoxyDate}{Date}
10/07/18
\end{DoxyDate}
Pumps out a data emitter and a rate struct

\begin{DoxyDate}{Date}
04/\+Jul/2018
\end{DoxyDate}
\begin{DoxyRefDesc}{Todo}
\item[\mbox{\hyperlink{todo__todo000032}{Todo}}]Output the X\+ML after being tokenized into an intermediate format for faster future parsing\end{DoxyRefDesc}


\begin{DoxyRefDesc}{Todo}
\item[\mbox{\hyperlink{todo__todo000033}{Todo}}]perahps create this as a seperate app that can be laucnhed from within the core/\+C\+LI \end{DoxyRefDesc}


\begin{DoxyDate}{Date}
04/\+Jul/2018
\end{DoxyDate}
When moving to child/sibling work exclusively on the top/current node and get the \char`\"{}next sibling\char`\"{}.

When the lower level child is done, return to the previous level above.

First child is node-\/$>$first\+\_\+node(). Subsequent nodes are node-\/$>$next\+\_\+sibling

\begin{DoxyDate}{Date}
03/\+Jun/2018

30/\+Jun/18
\end{DoxyDate}
The compiler will output many of this class. Each one will contain a variable that it will act on. These will be stored in a final \char`\"{}\+Data emitter\char`\"{} class that will be loaded by the protocol. The protocol will then make calls to it to get data from it.

\begin{DoxyDate}{Date}
04/\+Jul/2018 
\end{DoxyDate}
